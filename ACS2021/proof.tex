\documentclass[11pt,letterpaper]{article}
\usepackage{cogsys}
%\usepackage{cogsysapa} % deprecated!
% \usepackage{graphicx}
\usepackage[T1]{fontenc}
\usepackage{times}
\usepackage[pdftex]{graphicx} % use this when importing PDF files

% natbib required to produce author-year citations;
% apacite is not properly supported and may lead to errors
\usepackage{natbib}
\setlength{\bibsep}{0.75ex}

 % First page headings for accepted submissions.
\cogsysheading{X}{20XX}{1-6}{X/20XX}{X/20XX}
 % First page headings for poster submissions.
%\cogsysposterheading{First}{2012}{1-18}

\ShortHeadings{Epistemic Entrenchment and Counterfactual Reasoning \\
in Automated Theory Change}
              {Xue Li, Alan Bundy, Eugene Philalithis}


\usepackage[many]{tcolorbox}
\usepackage{amsmath}
\usepackage{amssymb}
\usepackage{proof,amsthm, colordvi,changebar,ifthen}
\usepackage{tabularx}
\usepackage{multirow}
\usepackage{makecell}
\usepackage{enumitem}
\usepackage{mathtools}
\usepackage{slashbox}
\sloppypar % get rid of overrun a line



%----
\newtheorem{defn}{Definition}[section]
\newtheorem{ex}{Example}
\newtheorem{theorem}{Theorem}[section]
\newtheorem{heuristic}{Heuristic}
\newtheorem{exa}{Example}[section]
\newtheorem{assumption}{Assumption}[section]
\newtheorem{spec}{Specification}
\newtheorem{post}{Postulate}
\newcommand{\pf}[1]{\mathcal{#1}(\mathbb{PS})}
\newcommand{\ps}{\mathbb{PS}}
\newcommand{\dis}[1]{$\mathcal{d}(#1)$}
\newcommand{\red}[1]{\textcolor{red}{#1}}
\newcommand{\blue}[1]{\textcolor{blue}{#1}}
\newcommand{\green}[1]{\textcolor{green}{#1}}
\newcommand{\cyan}[1]{\textcolor{cyan}{#1}}
\newcommand{\brown}[1]{\textcolor{brown}{#1}}
\newcommand{\theory}{\mathbb{T}}
\newcommand{\norm}{\mathcal{N}}
\newcommand{\signature}{\mathcal{S}}
\newcommand{\struc}{\mathbb{PS}}
\newcommand{\num}{\mathcal{N}}
\newcommand{\newtheory}{\mathbb{T'}}
\newcommand{\uneq}{$\setminus${=}}
\newcommand{\true}{{\cal T}(\struc)}
\newcommand{\false}{{\cal F}(\struc)}
\newcommand{\incompa}{{\cal IC}(\theory,\struc)}
\newcommand{\insuffa}{{\cal IS}(\theory,\struc)}
\newcommand{\incompb}{{\cal IC}(\nu_k(\theory),\struc)}
\newcommand{\insuffb}{{\cal IS}(\nu_k(\theory),\struc)}

\newtcolorbox{mybox}[1]
{
  colframe = {blue},
  colback  = {grey},
  coltitle = {black},  
  title    = {#1},
}


\begin{document} 

\title{Signature Entrenchment and Counterfactual Reasoning \\
in Automated Theory Change}
 
\author{Xue Li}{Xue.Shirley.Li@ed.ac.uk}
\author{Alan Bundy}{A.Bundy@ed.ac.uk}
\author{Eugene Philalithis}{E.Philalithis@ed.ac.uk}
\address{School of Informatics, University of Edinburgh, UK}
\vskip 0.2in
 
\begin{abstract}
This file contains the proofs of theorems in the main paper organised section by section.   
\end{abstract}

\section{Introduction}
\label{intro}

\noindent 
There is no theorems to prove in this section.

\section{Conceptual change}
\label{sec:conceptualChange}

\section{The ABC Repair System}
\label{sec:operation}
There is no theorems to prove in this section.


 There is no theorems to prove in this section.
\section{Measuring Signature Entrenchment}
\label{sec:ee:sig}
 
\begin{theorem}
\label{therem:ref:theog}
If there is no path from predicate $p$ to predicate $q$ in the theory's theory graph, then assertions of $p$ cannot contribute to any proof of an assertion of $q$.
\end{theorem}
\begin{proof}
In Theorem \ref{therem:ref:theog}, no path from predicate $p$ to predicate $q$ in the theory's theory graph means that there is no rule connection between $p$ and $q$. Therefore, assertions of $p$ and $q$ are independent from each other, so that assertions of $p$ cannot contribute to any proof of an assertion of $q$. 
\end{proof}

Therefore, theorem \ref{therem:ref:theog} decides whether adding a theorem of $p$ has an impact of building the proofs of $q$’s instances.

\subsection{Predicate Entrenchment}

\begin{theorem}\label{spec:pe}
Based on the  preferred distance, the important properties that predicate entrenchment $e(p)$ has are as follows, where $p$, $p_{1}$ and $p_{2}$ are predicates, and $\mathbb{S}_{p}$ is the set of predicates which occur in $\ps$ while $\mathbb{S}_{t}$ is the set of predicates which occur in the theory but not in $\ps$.
\begin{enumerate}
\item $\forall p \in (\mathbb{S}_{t} \cup \mathbb{S}_{p}),\ e(p)$ has exactly one value.\newline
\textnormal{The entrenchment of a predicate should be just one value.}


\begin{proof}

Based on Equation (3), the preferred distance of a predicate has only one value, accordingly, the entrenchment score of a predicate has one fixed value because its calculation function is monotonous. In summary, property 1 in Theorem \ref{spec:pe} is held by our measurement.
\end{proof}

\item $ 0 < e(p) \leq 1$. \newline
\textnormal{The range of an entrenchment should be [0,1], where 0 means that a predicate is not trusted at all and 1  represents that the predicate is most entrenched and fully trusted.}
\begin{proof}
Based on Equation (3) and (4), we can conclude that:
\begin{equation*}
    \forall pd(p_1) \neq \infty,\ 0\geq pd(p_1)\geq pd_{Max}
\end{equation*}
$\because \forall pd(p_1) \neq \infty,\ e(p_1) = 1- \frac{pd(p_1)}{pd_{Max}+1}$,\\
$\therefore 1- \frac{pd_{Max}}{pd_{Max}+1}  \leq e(p_1) \leq1- \frac{0}{pd_{Max}+1}$\\
Therefore, we can conclude that:
\begin{equation}\label{equ:11}
    0 < \frac{1}{pd_{Max}+1} \leq e(p_1)\leq 1,\ \forall pd(p_1) \neq \infty
\end{equation}


On the other hand, $\forall pd(p_2)= \infty$, we have $e(p_2) =\frac{1}{pd_{Max}+2}$.\\
$\because 0 \leq pd_{Max} $\\
$\therefore 2< pd_{Max}+2$. Thus, we have the follows.
\begin{equation}\label{equ:22}
0<e(p_2) \leq \frac{1}{2}
\end{equation}
\end{proof}
\item $\forall p_{2} \in \mathbb{S}_{p},\  e(p_{2})=1\ \wedge \  \forall p_{1}\in \mathbb{S}_{t},\  0< e(p_{1}) < 1$.\newline
\textnormal{Because $\ps$ is more trusted than the theory, a preferred predicate is most entrenched whose entrenchment is 1. Any predicate appearing only in the theory is believed to some extent, but less than a preferred predicate. Meanwhile, any predicate that occurs in the theory is considered to convey some information. Therefore, its entrenchment is bigger than 0 but smaller than 1.  }
\begin{proof}
 $\because \forall p_1 \in \mathbb{S}_t,\ pd(p_1) = 0$.\\
$\therefore e(p_1) = 1- \frac{0}{pd_{Max}+1}  = 1$\\

On the other hand, $\forall pd(p_2)\in \mathbb{S}_t,\ pd(p_2) > 0$, so $pd_{Max}> 0$, we have 
\begin{equation*}\label{equ:predEntr1}
    e(p_2) =
    \begin{cases}
    1- \frac{pd(p_2)}{pd_{Max}+1}<1- \frac{0}{pd_{Max}+1}<1,\  pd(p) \neq \infty\\
    \frac{1}{pd_{Max}+2}<\frac{1}{2}<1\ pd(p_2) = \infty
    \end{cases}
\end{equation*} 
\end{proof}

\item $\forall p_{1},p_{2}  \in \mathbb{S}_{t},\ e(p_{1}) > e(p_{2})$, iff $pd(p_{1}) < pd(p_{2})$.\newline
\textnormal{When neither predicate occurs in $\ps$, $p_{1}$  is more entrenched than $p_{2}$  if and only if $p_{1}$ is closer to preferred predicates in terms of its preferred distance. The smaller $pd(p_{1})$ is, the more impact on $\ps$ changing  $p_{1}$ will have. }
\begin{proof}
Consider different combinations between $p_1$ and $p_2$, we prove this property as follows.
\begin{itemize}
    \item  When $pd(p_1)\neq \infty, pd(p_2)\neq \infty$, if $e(p_1)>e(p2)$, then $1- \frac{pd(p_1)}{pd_{Max}+1}>1- \frac{pd(p_2)}{pd_{Max}+1}$.\\
 $\therefore \frac{pd(p_1)}{pd_{Max}+1}< \frac{pd(p_2)}{pd_{Max}+1}$\\
 $\therefore pd(p_1) < pd(p_2)$.
 
 \item When $pd(p_1)\neq \infty, pd(p_2)= \infty$,  if $e(p_1)>e(p2)$, then $1- \frac{pd(p_1)}{pd_{Max}+1}>1- \frac{1}{pd_{Max}+1}$.\\
 $\therefore pd(p_1) < 1$, Then it can be concluded that $pd(p_1) < 1<pd(p_2)= \infty$
 \item The case that $pd(p_1)= \infty, pd(p_2)\neq \infty$ and $e(p_1)>e(p_2)$ does not exist.\\
 $\because e(p_2)= 1- \frac{pd(p_1)}{pd_{Max}+1}=\frac{pd_{Max}+1 - pd(p_1)}{pd_{Max}+1}$, and $pd_{Max} \geq pd(p_1)$, then $pd_{Max}- pd(p_1)\geq 0$.\\
 $\therefore  e(p_2)\geq \frac{1}{pd_{Max}+1}=e(p_1)$. Thus, $e(p_1)>e(p_2)$ cannot happen in this case. 
 
 \item The case that $pd(p_1)= \infty, pd(p_2)= \infty$ and $e(p_1)>e(p_2)$ does not exist because $e(p_1)=e(p-2)$.
\end{itemize}
In conclusion, $e(p_1)>e(p_2)$ only happen when $pd(p_1)\neq\infty$. We have proved that:  $e(p_1)>e(p_2) \implies pd(p_1)<pd(p_2)$.



On the other hand, if $pd(p_1)<pd(p_2)$, then neither of them is infinity. \newline
$\because e(p_1) = 1- \frac{pd(p_1)}{pd_{Max}+1}$, and $e(p_2) = 1- \frac{pd(p_2)}{pd_{Max}+1}$.\\
$\therefore e(p_1)-e(p_2) = \frac{pd(p_2)-pd(p_1)}{pd_{Max}+1}>0$\\
In summary, $pd(p_1)<pd(p_2) \implies e(p_1)>e(p_2) $
 
\end{proof}
\end{enumerate}
\end{theorem}




\subsection{Argument Entrenchment}
\label{sec:ee:args}

\noindent 
There is no theorems to prove in this section.

\section{Evaluation}
\label{sec:eva}

\noindent 
There is no theorems to prove in this section.


\section{Conclusion}
\label{conclusion}

\noindent 
There is no theorems to prove in this section.

\end{document} 

